%-------------------------------------------------------------------------------
%	SECTION TITLE
%-------------------------------------------------------------------------------
\cvsection{Proyectos}


%-------------------------------------------------------------------------------
%	CONTENT
%-------------------------------------------------------------------------------
\begin{cventries}

%---------------------------------------------------------
  \cventry
    {\href{https://www.github.com}{Enlace a github}} % Project URL
    {SmartU - Plataforma web de difusión y creación de ideas} % Project title
    {Diversas tecnologías web} % Project programming language
    {Sep. 2016 - Nov. 2017} % Date(s)
    {
      \begin{cvitems} % Description(s) of tasks/responsibilities
        \item {Trabajo de Fin de Grado bajo el marco del proyecto \textbf{La universidad conectada a la ciudad sostenible}.}
        \item {Sistema web donde las personas pueden publicar sus ideas y encontrar a gente que quiera participar en ellas.}
        \item {Tiene como objetivo \textbf{fomentar los proyectos en equipo} entre estudiantes de diferentes facultades.}
        \item {Contará con una \textbf{interfaz accesible y adaptable} a todo tipo de dispositivos, permitiendo una gran interacción con el usuario.}
      \end{cvitems}
    }

%---------------------------------------------------------
  \cventry
    {Proyecto de asignatura} % Project URL
    {Chirper} % Project title
    {HTML, CSS, PHP, JS y MySQL} % Project programming language
    {Feb. 2017 - Jun. 2017} % Date(s)
    {
      \begin{cvitems} % Description(s) of tasks/responsibilities
        \item {Proyecto final de la asignatura \textbf{Programación Web}.}
        \item {Permite el registro de usuarios y la publicación de entradas, y pueden recibir comentarios.}
        \item {Uso de aspectos innovadores de diseño en CSS3 como \textbf{Flexbox y Grid}.}
        \item {Interfaz \textbf{responsive} siguiendo la idea de diseño \textbf{mobile first}.}
      \end{cvitems}
    }

%---------------------------------------------------------
  \cventry
    {Proyecto de asignatura} % Project URL
    {Restaurants Manager} % Project title
    {Django y Bootstrap} % Project programming language
    {Sep. 2016 - Feb. 2017} % Date(s)
    {
      \begin{cvitems} % Description(s) of tasks/responsibilities
        \item {Proyecto final de la asignatura \textbf{Desarrollo de Aplicaciones para Internet}.}
        \item {Web creada con Django, aspecto visual realizado con Bootstrap y desplegada en Heroku.}
        \item {Gestión sencilla de restaurantes alojados en una base de datos \textbf{MongoDB}.}
        \item {Web totalmente \textbf{internacionalizada y traducida} a español e inglés.}
      \end{cvitems}
    }

%---------------------------------------------------------
  \cventry
    {Proyecto de asignatura} % Project URL
    {SCACV} % Project title
    {Java} % Project programming language
    {Abr. 2016 - Jun. 2016} % Date(s)
    {
      \begin{cvitems} % Description(s) of tasks/responsibilities
        \item {Proyecto final de la asignatura \textbf{Desarrollo de Software}.}
        \item {Applet de Java con la interfaz de un panel de control de un vehículo para controlar su velocidad.}
        \item {Proyecto realizado mediante el uso de \textbf{patrones de diseño}.}
        \item {Realización de \textbf{pruebas unitarias y de integración} para verificar el correcto funcionamiento.}
      \end{cvitems}
    }

%---------------------------------------------------------
  \cventry
    {Proyecto de asignatura} % Project URL
    {Napakalaki} % Project title
    {Java y Ruby} % Project programming language
    {Sep. 2014 - Jun. 2015} % Date(s)
    {
      \begin{cvitems} % Description(s) of tasks/responsibilities
        \item {Proyecto final de la asignatura \textbf{Programación y Diseño Orientado a Objetos}.}
        \item {Adaptación del famoso juego de cartas Munchkin a un programa de ordenador.}
        \item {Aplicación Java con interfaz hecha en Java Swing.}
        \item {Aplicación Ruby con interfaz en modo texto.}
      \end{cvitems}
    }

%---------------------------------------------------------
\end{cventries}
