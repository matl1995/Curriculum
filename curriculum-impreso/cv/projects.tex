%-------------------------------------------------------------------------------
%	SECTION TITLE
%-------------------------------------------------------------------------------
\cvsection{Proyectos}


%-------------------------------------------------------------------------------
%	CONTENT
%-------------------------------------------------------------------------------
\begin{cventries}

%---------------------------------------------------------
  \cventry
    {\href{https://github.com/matl1995/TFG}{Proyecto fin de grado}} % Project URL
    {Juego de tablero aumentado utilizando ARCore} % Project title
    {Android, Unity, ARCore} % Project programming language
    {Sep. 2017 - Sep. 2018} % Date(s)
    {
      \begin{cvitems} % Description(s) of tasks/responsibilities
        \item {Proyecto de investigación y desarrollo sobre tecnologías de \textbf{Realidad Aumentada} y el motor de videojuegos \textbf{Unity}.}
        \item {Centrado en la plataforma \textbf{ARCore}.}
        \item {Planificación y desarrollo llevado a cabo haciendo uso de \textbf{metodologías de desarrollo ágil}.}
      \end{cvitems}
    }

%---------------------------------------------------------
  \cventry
    {\href{https://github.com/matl1995/DAI}{Proyecto de asignatura}} % Project URL
    {Gestor de restaurantes} % Project title
    {Django y Bootstrap} % Project programming language
    {Sep. 2017 - Ene. 2018} % Date(s)
    {
      \begin{cvitems} % Description(s) of tasks/responsibilities
        \item {Proyecto final de la asignatura \textbf{Desarrollo de Aplicaciones para Internet}.}
        \item {Web creada con \textbf{Django} e interfaz de usuario realizada con \textbf{Bootstrap}.}
        \item {Gestión sencilla de restaurantes alojados en una base de datos \textbf{MongoDB}.}
      \end{cvitems}
    }

%---------------------------------------------------------
  \cventry
    {\href{https://github.com/matl1995/PW}{Proyecto de asignatura}} % Project URL
    {DEC - La red social} % Project title
    {HTML, CSS, PHP, JS y MySQL} % Project programming language
    {Feb. 2017 - Jun. 2017} % Date(s)
    {
      \begin{cvitems} % Description(s) of tasks/responsibilities
        \item {Proyecto final de la asignatura \textbf{Programación Web}.}
        \item {Web centrada en el funcionamiento de una red social.}
        \item {Permite el registro de usuarios y la publicación de entradas, y pueden recibir comentarios.}
      \end{cvitems}
    }

%---------------------------------------------------------
  \cventry
    {\href{https://github.com/matl1995/PDM/tree/master/Museo}{Proyecto de asignatura}} % Project URL
    {Museo interactivo QR} % Project title
    {Android} % Project programming language
    {Feb. 2017 - Jun. 2017} % Date(s)
    {
      \begin{cvitems} % Description(s) of tasks/responsibilities
        \item {Proyecto final de la asignatura \textbf{Programación de Dispositivos Móviles}.}
        \item {Aplicación Android de lectura de códigos \textbf{QR}.}
        \item {Al leer el código de una obra de arte de un museo, presenta al usuario un trivial de preguntas relacionadas con ella.}
      \end{cvitems}
    }

%---------------------------------------------------------
\end{cventries}
